% CS 455, SP'24 Configuration Management Plan template
% Software design template based on the template from
% https://tex.stackexchange.com/questions/42602/software-requirements-specification-with-latex
%
\documentclass[letterpaper,12pt,oneside,listof=totoc]{scrreprt}
\usepackage{listings}
\usepackage{underscore}
\usepackage[bookmarks=true]{hyperref}
\hypersetup{
  %  bookmarks=true,                                % show bookmarks bar
    pdftitle={Configuration Management Plan},       % title
%    pdfauthor={Yiannis Lazarides},                  % author
%    pdfsubject={TeX and LaTeX},                     % subject of the document
%    pdfkeywords={TeX, LaTeX, graphics, images},     % list of keywords
    colorlinks=true,                                % false: boxed links; true: colored links
    linkcolor=blue,                                 % color of internal links
    citecolor=black,                                % color of links to bibliography
    filecolor=black,                                % color of file links
    urlcolor=purple,                                % color of external links
    linktoc=page                                    % only page is linked
}%
\def\myversion{1.0 }

\date{\today}
\author{} % suppress warning, do not fill this in
\begin{document}

% we don't use \maketitle because we overide the default title page here
\begin{titlepage}
\flushright
\rule{\textwidth}{5pt}\vskip1cm
\Huge{CONFIGURATION MANAGEMENT PLAN}\\
\vspace{1.5cm}
for\\
\vspace{1.5cm}
Risk Management System\\
\vspace{1.5cm}
\LARGE{Release 0.1\\}
\vspace{1.5cm}
%\LARGE{Version \myversion approved\\}
\vspace{1.5cm}
Prepared by Software Risk Tracking Team\\
\vfill
\rule{\textwidth}{5pt}
\end{titlepage}

\tableofcontents
% this will be automatically created from chapters, sections, and subsections

\listoffigures
% this will be automatically created from the figure environment

\listoftables
% this will be automatically created from the table environment

\chapter*{Revision History}
% Update this table for each revision of the requirements
% Add the new content followed by a \hline

\begin{tabular}{| c | p{0.30\textwidth} | p{0.60\textwidth} |}
\hline
Date     & Description   & Revised by \\
\hline
3/6/24 & Baseline & Software Risk Tracking CM Team \\
\hline
\end{tabular}

% What is a Configuration Management Plan (CMP)?
%
% ``Configuration management encompasses the technical and administrative activities concerned with the creation, maintenance, controlled change and quality control of the scope of work.
% 
% A configuration is the functional and physical characteristics of a product as defined in its specification and achieved through the deployment of project management plans.''[1]
% 
% [1]Definition from APM Body of Knowledge 7th edition
% 
% This document was created from a CM Plan template used by NASA's Independent Verification and
% Validation (IV&V) program.

\chapter{INTRODUCTION}

\section{Purpose}

% Identify the purpose of this CMP and its intended audience. (address who should read this and why)
This document outlines the plan for Configuration Management (CM) for the Software Risk Tracking product. This document is intended for use by developers on the Software Risk Tracking Project. This document will keep track of how changes to the source code are managed, as well as processes for requesting changes, and who determines whether or not changes can be made.
\begin{table}[htbp]
\centering
\caption{Definitions}
\begin{tabular}{|p{2.5in}|p{2.5in}|}
    \hline
    Baseline & A minimum or starting point used for comparisons\\
    \hline
    Issue & A problem or challenge that arises during the CM process\\
    \hline
    Build & The process of assembling software components and source code into an executable or deployable form\\
    \hline
    CI & Configuration Items\\
    \hline
\end{tabular}
\label{tab:definitions}
\end{table}

\section{References}

% This section is optional. List any documents, if any, which were used as sources of information.
Configuration Management Baseline Definitions: \\
\indent https://cmpic.com/whitepapers/configuration-management-baselines.htm

\chapter{CONFIGURATION MANAGEMENT}

% Give a general description of the purpose of CM in the project.

\section{Organization}

% Describe how the CM function is structured in the project.
The CM function within the project will be to control how changes are made to the project, when and how releases are made, and how the changes and releases will be tracked. The CM Plan should be referenced by anyone working as a detailed description regarding how changes to the project are handled. Changes include merges into the master branch and any issues/bugs. 

\section{Responsibilities and Applicability}

% Describe who must follow the CMP and who coordinates/leads the CM effort.
The CM Plan should be followed by anyone involved in the development process for the Software Risk Tracking project. The CM effort is being lead by the CM/SQA team, which is responsible for maintaining this plan as well as providing quality assurance for the software being produced and merged into the master branch. 

\chapter{CONFIGURATION MANAGEMENT ACTIVITIES}

% This section details the specific procedures and activities used by the project team to control work products.

\section{CI Identification}

 %How are CI's identified (numbers, hashes, etc.)? List the types of CIs that must be tracked (e.g. team meeting notes, requirements, training info, source code, test scripts, test results, design docs, test docs, builds, releases, CM tools, developer tools, reviews, etc.).



\subsection{Software Development Library}

% How will CIs be stored and organized so that team members and stakeholders can access them? Detail the control mechanism(s), number of libraries, backup and disaster recovery plans and procedures, retention policy and plan (what needs to be archived, for who, and for how long), and where the SDL will be stored.
The CIs being used for this product include the following:
\begin{itemize}
    \item CS Server, which will house and run the program
    \item GiTea, which will be used for version control and as a backup 
    \item The SRS, CM Plan, and SQA Plan will all be stored in the Resources folder in GiTea, along with any other necessary documentation.
    \item A work log will be kept to document the number of hours spent by each team member on the project and what was being done during that time. This will also be stored in the Resources folder.
\end{itemize}

\subsection{Project Baselines}

% What are the planned project baselines? List how and when each baseline will be created, who authorizes the baseline, who verifies the baseline, the purpose of the baseline, and what is in the baseline.
UML Representation:

1. Initial UML Modeling: Begin by creating UML diagrams that represent the software's architecture, design, and key components.

2. Capturing Requirements: Ensure that the UML diagrams accurately capture the requirements and functionalities of the software.\\

3. Stakeholder Review: Conduct reviews with stakeholders to validate the UML representation and ensure alignment with their expectations and needs.\\\\Construction Phase:

1. Code Implementation: Translate UML models into code through software development.

2. Version Control: Utilize Git to manage changes to the code base and track its evolution.

3. Continuous Integration: Implement continuous integration practices to ensure that code changes integrate smoothly and do not introduce regressions.\\\\Reviews:

1. Code Reviews: Conduct regular code reviews to assess the quality, maintainability, and the adherence to coding standards of the implemented code.

2. Design Reviews: Periodically review the UML representation and its alignment with the evolving software requirements and architecture.

3. Iterative Feedback: Incorporate feedback from reviews into the development process to refine the UML representation and improve the code base.

\subsection{Software Builds}

% How will a specific build be identified and controlled? What is included in a build? If there will be different types of builds (for example, development, candidate, release), describe information for each type.
At this time, the build process is still to be determined.

% Docker builds images by reading the instructions from a Dockerfile. A Dockerfile is a text file containing instructions for building your source code.\\\\ Building for Release and Testing:
% \par For release builds:
% \par \hspace{\parindent} Use your Dockerfile.release to build an image optimized for production use. This image should only include necessary production dependencies and code.
% For testing: Use your Dockerfile.test to build an image suitable for running tests. This image may include additional dependencies or configurations required for testing purposes.


% docker image? release won't include test code? separate build for release and test?

\section{Configuration Control}

\subsection{Tools for Change Control}

% List the software tools used for CM and describe how a project member or stakeholder get access to the tool. Where are the tools housed/hosted?
This project will use Git. Git is a version control tool used in software development. Git is a tool for tracking changes in software, allowing several engineers to collaborate in a non-linear fashion.

\subsection{Procedure for Change Requests}

% Who can request changes? Who reviews change requests? Who approves change requests?
1. Request changes: Anyone involved in the project, like developers or admins, can request changes. Requests can be submitted through GiTea issues.

2. Change Request Reviews: Requests are reviewed by at least two team leads. This will include a checklist that ensures that the change will not have an adverse effect on the project.

3. Change Request Approval: After review, change requests can be approved or denied by team leads.


\subsection{Procedure for Changing Baselines}

% Who can request baseline changes? Who reviews baseline change requests? Who approves baseline change requests?
1. Baseline Changes: Baseline modifications can be requested by developers and customers/stakeholders. These requests can be made either through GiTea also.

2. Reviews Baseline Changes: Requests for baseline changes shall be reviewed by the CM Team and team leads. The review process shall involve looking into how the baseline change will affect the schedule and design of the project.

3. Approves Baseline Changes: Requests shall be approved or denied by team leads upon proper review.

\subsection{Review Procedure}

% List when software reviews are held, how they are held, how notes are recorded, how issues are tracked to completion, and where notes and issues are recorded.
Software reviews will take place when a merge request is created or when a CR is made. These reviews can be done by all developers, and any notes will be commented under the merge request. This will also handle how issues found during a review are tracked. All notes and issues will be recorded inside of GiTea.

Upon completion of a review, the changes shall be merged to the dev branch, and unit tests will be written if necessary. If the tests show any issues that were not already caught, an issue will be created in GiTea and the appropriate team will be notified. The tests will follow the same review and merge procedures, but will not be included in the release package.

\section{Status Accounting and Reporting}
This section defines the type of events to be reported and accounted for, who they are reported to, how they are reported and recorded for documentation purposes, and the process that is to be expected upon a release.

\subsection{Status Reporting}

% List the information that must be reported, who it is reported to, and how the information is controlled. This list should include periodic stats reports within the project, reports to management, and reports to stakeholders.

Among the three sub-teams collaborating on this project (View, Model, and CM/SQA), any moderate issues that may arise throughout the development of the project that could hinder the release of work products by the intended deadline (i.e. server errors preventing software from working as intended, difficulty in creating code for software increments that will meet stakeholders' expectations, or testing processes result in more frequent than expected software send-backs) must be reported to the sub-team leader of the group affected by the issue. % Refer to table for list of these leaders. 
The leaders can then in turn report to stakeholders in the event that a development increment cannot be completed by the intended deadline, if necessary. These reports should be made at least 24 hours before an intended release so that notifications can be sent out to the respective members in advance.\\
\\
It is the responsibility of not just the leaders but also every single team member to properly communicate any significant status updates in a timely manner though the favored use of Microsoft Teams in order to maintain the effective control of information throughout the lifespan of the project. Microsoft Teams allows the team as a whole to have one to two weekly meetings to report status updates, however, at any point in time any team member can post status updates to the chat log in between meetings.

\subsection{Issue Tracking}

% Describe how issues (tickets) will be recored, tracked, and reported. Describe who will manage issue assignment and completion as well as who will ensure (audit) all issues are handled.
For documentation purposes, tickets in regards to issues that may occur throughout the software engineering process must title in the format x-x.x.x, where x- refers to the number corresponding to the team who issued it (1: View, 2: Model, 3: CM/SQA), and x.x.x refers to the issued date. Tickets must also include a brief description of the issue/event that occurred. A log of these tickets can be compiled within a single text document accessible to any member. \\
% Example form for ticket
\\
Sub-team leaders can assess and manage these issues and communicate with the sub-team members affected by the issue in order to resolve them. The CM team will persistently monitor the log for newly created tickets to ensure they become resolved in a timely manner. This process involves the team leaders meeting with either all of the affiliated sub-team members or just the member that created the ticket to discuss in greater detail the issue being presented. If the issue can be resolved in a relatively short amount of time between members and leaders, then presented solution can be resubmitted for review. However, if the issue can not be resolved, for example by the next release date, then the affiliated team should meet with the management team to discuss a temporary solution or allow stakeholders to be notified of the potential setback.

\subsection{Release Process}

% Describe the release process. The process should detail what is included in a release, when releases are provided and to who, the media for the release, how any known problems or fixes in the release are reported, and release installation instructions.
Each week a new release will be made. This will follow a strict schedule, allowing developers to plan ahead for the tasks needing to be completed for each release. These releases will be tracked by Milestones in GiTea. Each Saturday, a weekly checklist will be sent out to developers with the tasks that need to be completed before the next release. Developers will have until Wednesday at 11:59 P.M. CST to complete the assigned tasks or discuss with their team lead and the CM team the issue(s) they are having and their plan to resolve it. 

By Thursday at 11:59 P.M. CST, all necessary unit tests for new source code are to be submitted. Reviews should be completed and the release ready by Friday at 11:95 P.M. CST.

Any issues that do not get resolved by the release will be moved to the next week, but shall be resolved in a timely manner as to not hinder the completion of new tasks. 

\section{CM Milestones}

% List all the project milestones (e.g. baselines, reviews, reports, releases). Describe how the milestones will be tracked and reported. Specify the criteria associated with each listed milestone.
Milestones will be set and tracked through GiTea by team leads. These will be followed at closely as possible to ensure completion of the program by the date specified by stakeholders.

\subsection{Baselines}
Baselines for each release will be listed alongside the respective Milestone in GiTea. These baselines will outline what is expected to be completed for the next release.

\subsection{Design Review}
Design review will ensure that the software is meeting the requirements. This will initially be done before any software is written, but subsequent reviews will take place in the event that any changes are proposed to the design.

\subsection{Development Milestones}
Similarly to Baselines, Development Milestones will be tracked in GiTea. It is the responsibility of team leads to make sure their team is on track and resolving any issues that need to be addressed. Development Milestones will follow the schedule laid out in the 3.3.3.

\subsection{Testing Milestones}
Testing milestones will be tracked from reports. As stated in 3.3.3, tests will also be released on a schedule to ensure timely completion. This will allow any issues that arise to be made known to the appropriate team. At a minimum, unit testing, integration testing, and acceptance testing will be performed. 

\subsection{Deployment Milestones}
See 3.3.3

\subsection{Release Milestones}
See 3.3.3


\section{Training}

% List the types and amounts of training required for the development team. Include how the training status will be tracked and reported.
Training included in-person lecture on Git, the Wt library, CMake, and GTest. It is also the responsibility of team members to educate themselves further on any other necessary libraries or tools used during the development process. 
\end{document}
